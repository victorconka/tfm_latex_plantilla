%============================================================================================
% DATOS DE LA TESIS
% 1.RELLENAR LOS DATOS
\newcommand{\titulo}{[Título de Trabajo Fin de Estudios]}
\newcommand{\autor}{[Nombre del Alumno]}
\newcommand{\director}{[Nombre del Director de Proyecto]}
\newcommand{\fecha}{\today}
\newcommand{\ciudad}{[Ciudad del Centro de Estudios]}
\newcommand{\universidad}{[Centro de Estudios]}
\newcommand{\escuela}{[Escuela del Centro de Estudios]}
\newcommand{\master}{[Nombre de Estudios (Máster/Grado)]}
\newcommand{\especialidad}{[Modalidad de Estudios]}
\newcommand{\tipoTesis}{[Tipo de Trabajo]}
\newcommand{\palabrasClave}{[Palabras, Clave, Rellenar]}
%============================================================================================
% REQUISITOS TFE
% 2. MODIFICAR SI FUERA NECESARIO
\newcommand{\papel}{a4paper} % tamaño papel - a4
\newcommand{\letra}{12pt} % letra - 12
\newcommand{\interlineado}{1.5} %interlineado 1.5
\newcommand{\margenes}{25mm} % margenes 25mm cada uno
\newcommand{\distribucion}{twoside} %papel a doble cara
% texto justificado \justify
% encabezado o pie de página
% numero de pagina
%============================================================================================
\documentclass[\papel, \letra, \distribucion]{article} %tipo documento y tamaño letra
\usepackage[\papel, headsep=\letra, top=\margenes, bottom=\margenes, left=\margenes, right=\margenes]{geometry}
\usepackage{ragged2e} %\justify
% cabecera y pie de páginas----------------------
% 3. Modificar posición de cabecera y pie de página al gusto
\usepackage{fancyhdr}
\pagestyle{fancy}
\fancyhf{}
\fancyhead[LE,RO]{}  %left even, right odd
\fancyhead[RE,LO]{}     %right even, left odd
\fancyhead[CE,CO]{\titulo} %center even/odd
\fancyfoot[RE,LO]{}
\fancyfoot[LE,RO]{\thepage}
\fancyfoot[CE,CO]{}
\renewcommand{\headrulewidth}{1pt}
\renewcommand{\footrulewidth}{1pt}
\setlength{\headheight}{15pt}% asignamos espacio para la cabecera
% -----------------------------------------------
% fondo de páginas
% 4. Configurar al gusto la marca de agua del documento
\usepackage{background} %fondo de páginas
\backgroundsetup{contents={}} %deshabilitar cualquier configuración por defecto
\newcommand{\setbackground}{
    \backgroundsetup{
        scale = 1.7,
        color = yellow,
        opacity = 0.03,
        vshift = -7cm,
        hshift = -0.5cm,
        angle = 48,
        contents = {
            \includegraphics[
                width = \paperwidth,  
                height = \paperheight, 
                keepaspectratio
            ]{logo.png}
        }
    }
}

\usepackage[utf8]{inputenc}
\usepackage[spanish]{babel} %configuramos idioma de corrección ortográfica español
\usepackage{csquotes} %tildes y acentos
\usepackage{pdfpages} %portada/contraportada pdf
\usepackage{biblatex} %bibliografía
\addbibresource{mybib.bib} %Import the bibliography file
\usepackage{setspace} %paquete para interlineado \setstrech
\usepackage{afterpage} %comando para aplicar algo a partir de sig. pág.
\usepackage{booktabs} %top/bottom rules

% enlaces en la tabla de contenidos---------------------------------
\usepackage{color}   %May be necessary if you want to color links
\usepackage{hyperref}
\hypersetup{
    colorlinks=true,  %set true if you want colored links
    linktoc=all,      %set to all if you want both sections and subsections linked
    linkcolor=black,  %choose some color if you want links to stand out
}
% -----------------------------------------------
% pagina en blanco de relleno
\newcommand\blankpage{
    \null
    \backgroundsetup{contents={}}
    \thispagestyle{empty}%
    \addtocounter{page}{-1}%
    \newpage
    }
%------------------------------------------------

% funciones de relleno se pueden borrar
\usepackage{lipsum}% for generating filler text
\newcommand{\f}[1]{
    \begin{figure}[ht]
        \centering
        \includegraphics[width=8cm]{example-image-a}
        \caption{Example image {#1} }
    \end{figure}
}
\newcommand{\tbl}[1]{
    \begin{table}[ht]
        \centering
        \begin{tabular}{ c c c } \toprule
        \textbf{A} & \textbf{B} & \textbf{C} \\ \bottomrule
         cell1 & cell2 & cell3 \\ 
         cell4 & cell5 & cell6 \\  
         cell7 & cell8 & cell9 \\ \bottomrule
        \end{tabular}
        \caption{Example table {#1} }
    \end{table}
}
%--------------------------------------------------------

\begin{document}
\pagenumbering{gobble} %deshabilitar numeración
\setstretch{\interlineado} %interlineado

% PORTADA
\includepdf{sections/cover/TFM_TAII_front.pdf}
\afterpage{\blankpage} %relleno con página en blanco

% PORTADA INTERIOR
\title{\titulo}
\author{\autor}
\date{\fecha}

%--------------------------------------------------------
%\maketitle
\begin{titlepage}
    \begin{center}
    
        {\Large \bf \tipoTesis}\\
        \vspace{1cm}
        {\Large \normalfont \bfseries \titulo}\\ 
        
        \vspace{1.5 cm}
        \includegraphics[height=5cm]{logo.png}
        \vspace{1.5 cm}
        
        {\Large \bf \universidad}\\
        \vspace{2mm}
        {\Large \bf \escuela}\\
        \vspace{2cm}
        
        {\bf \master}\\
        {\bf \especialidad}\\
        \vspace{2cm}
        
        {\bf ALUMNO}\\
        {\large \autor}\\
        \vspace{0cm}
        
        {\bf DIRECTOR}\\
        {\large \director}\\
        \vspace{2cm}
        
        {\ciudad, \fecha}
    \end{center}
\end{titlepage}
\afterpage{\blankpage} %relleno con página en blanco

% DEDICATORIA
\include{sections/01-dedication}
\afterpage{\blankpage}%relleno con página en blanco

% configuración de fondo de páginas
\setbackground

% ÍNDICE
\pagenumbering{roman} %numeración romana para índices

\tableofcontents    %índice de contenido
\listoffigures      %índice de figuras
\listoftables       %índice de tablas
\let\cleardoublepage\clearpage

\justify
% ABSTRACT
\pagenumbering{arabic} %numeración para contenido
\section{Abstract}
\label{section:abstract}

\lipsum[1] esto es una cita\cite{DummyArticle:2}. \newline
\lipsum[2] Algo mas de texto.

\vfill
\textbf{Keywords: } \textit{\palabrasClave}

% INTRODUCCIÓN
\section{Introducción}
\label{section:introduction}
\lipsum[1]\cite{DummyArticle:1}
\lipsum[2]\cite{DummyBook:1}

% MATERIALES Y MÉTODOS
\section{Materiales y Métodos}
\label{section:materials-and-methods}

\lipsum[1-3]

% RESULTADOS Y DISCUSIÓN
\section{Resultados y Discusión}
\label{section:results}

\lipsum[1-2]
\f{1}
\lipsum[3]
\f{2}
\lipsum[4]
\tbl{1}
\lipsum[5]
\tbl{2}
\lipsum[6]
\tbl{3}

% CONCLUSIONES
\section{Conclusiones}
\label{section:concluciones}
\lipsum[1-2]

% BIBLIOGRAFÍA
% filtro de bibliografía
\defbibfilter{articleinproceeding}{
  \type{inproceedings} \or \type{article}
}
%--------------------------------------------------------------------------

\section{Bibliografía}
\label{section:references}

\printbibliography[filter=articleinproceeding,title={Artículos}]
\printbibliography[type=book,title={Libros}]
\let\cleardoublepage\clearpage

% CONTRAPORTADA
\backgroundsetup{contents={}}
\includepdf{sections/cover/TFM_TAII_back.pdf}
\end{document}
